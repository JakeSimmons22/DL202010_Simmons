% Digital Logic Report Template
% Created: 2020-01-10, John Miller

%==========================================================
%=========== Document Setup  ==============================

% Formatting defined by class file
\documentclass[11pt]{article}

% ---- Document formatting ----
\usepackage[margin=1in]{geometry}	% Narrower margins
\usepackage{booktabs}				% Nice formatting of tables
\usepackage{graphicx}				% Ability to include graphics

%\setlength\parindent{0pt}	% Do not indent first line of paragraphs 
\usepackage[parfill]{parskip}		% Line space b/w paragraphs
%	parfill option prevents last line of pgrph from being fully justified

% Parskip package adds too much space around titles, fix with this
\RequirePackage{titlesec}
\titlespacing\section{0pt}{8pt plus 4pt minus 2pt}{3pt plus 2pt minus 2pt}
\titlespacing\subsection{0pt}{4pt plus 4pt minus 2pt}{-2pt plus 2pt minus 2pt}
\titlespacing\subsubsection{0pt}{2pt plus 4pt minus 2pt}{-6pt plus 2pt minus 2pt}

% ---- Hyperlinks ----
\usepackage[colorlinks=true,urlcolor=blue]{hyperref}	% For URL's. Automatically links internal references.

% ---- Code listings ----
\usepackage{listings} 					% Nice code layout and inclusion
\usepackage[usenames,dvipsnames]{xcolor}	% Colors (needs to be defined before using colors)

% Define custom colors for listings
\definecolor{listinggray}{gray}{0.98}		% Listings background color
\definecolor{rulegray}{gray}{0.7}			% Listings rule/frame color

% Style for Verilog
\lstdefinestyle{Verilog}{
	language=Verilog,					% Verilog
	backgroundcolor=\color{listinggray},	% light gray background
	rulecolor=\color{blue}, 			% blue frame lines
	frame=tb,							% lines above & below
	linewidth=\columnwidth, 			% set line width
	basicstyle=\small\ttfamily,	% basic font style that is used for the code	
	breaklines=true, 					% allow breaking across columns/pages
	tabsize=3,							% set tab size
	commentstyle=\color{gray},	% comments in italic 
	stringstyle=\upshape,				% strings are printed in normal font
	showspaces=false,					% don't underscore spaces
}

% How to use: \Verilog[listing_options]{file}
\newcommand{\Verilog}[2][]{%
	\lstinputlisting[style=Verilog,#1]{#2}
}




%======================================================
%=========== Body  ====================================
\begin{document}

\title{ELC 2137 Lab 3: Adder}
\author{Jake Simmons and Haonan Jin}

\maketitle


\section*{Summary}

In this lab, we first built a half adder and with it drew a schematic and wiring diagram. Then we tested it to see if its behaviour matched the half adder truth diagram. Next we built a full adder and with it drew a schemtaic and wiring diagram. Then we tested this one as well to see if its behaviour matched the full adder truth diagram. Lastly, we created a two bit adder. We did this by combining two full adders then checked to see if its behaviour matched the two bit adder truth table.  


\section*{Q\&A}

\begin{enumerate}
	\item Which gates could we use for combining the carry bits?
	\begin{enumerate}
		\item The gates we could use for combining the carry bits are XOR, AND Gates.
	\end{enumerate}
    \item Which one should we use and why?
    \begin{enumerate}
    	\item We should use the XOR Gate, because in binary when you add two ones together, the output will give you a zero with an one carried over. A XOR Gate, when inputted with two ones the output is zero which is same when you add two ones in binary. Also in a XOR Gate if one and zero is inputted a one is outputted which is the same when you add a one and zero in binary. If we were to use the AND Gate to add two binary numbers, one plus one would output one which in binary is not correct.
    \end{enumerate}
\end{enumerate}



\section*{Results}

\begin{center}
	\caption{Table1: Proof carry outputs of the first and second stage HAs cannot be high at the same time}
	\begin{tabular}{ccc|ccc|cc}
		\toprule
		Cin&A& B & C1 & S1 & C2 & Cout & S \\
		\midrule
		0&0&0&0&0&0&0&0 \\
		0&0&1&0&1&0&0&1 \\
		0&1&0&0&1&0&0&1 \\
		0&1&1&1&0&0&1&0 \\
	    1&0&0&0&0&0&0&1 \\
		1&0&1&0&1&1&1&0 \\
		1&1&0&0&1&1&1&0 \\
		1&1&1&1&0&0&1&1 \\
		\bottomrule
		\end{tabular}\medskip
	\end{center}
\begin{center}
	\begin{figure}
		\includegraphics[width=1\textwidth]{Scan_1.pdf}
		\caption{Schematics and Wiring Diagram of the Half and Full Adder}
		\end{figure}
	\end{center}
\begin{center}
	\begin{figure}
		\includegraphics[width=.6\textwidth]{Image.jpeg}
		\caption{Picture of the Half Adder}
	\end{figure}
\end{center}
\begin{center}
	\begin{figure}
		\includegraphics[width=.6\textwidth]{Image1.jpeg}
		\caption{Picture of the Full Adder}
	\end{figure}
\end{center}
\begin{center}
	\begin{figure}
		\includegraphics[width=.6\textwidth]{Image2.jpeg}
		\caption{Picture of the 2-Bit Adder}
	\end{figure}
\end{center}

\clearpage
\section*{Code}



\end{document}
